\documentclass[10pt]{article}
\usepackage{booktabs}
\usepackage{fullpage}
\usepackage{multirow}

\title{The phpWatch 2 Developer API}
\author{Aaron M. Rosenfeld}
\begin{document}
\maketitle

\section{Introduction}
This document provides developers with the information necessary to develop full featured, well integrated extensions for
phpWatch 2.  Nearly all of the externally available functionality is addressed in an attempt to educate developers about
the full potential of their extensions.

\section{Terminology}
It is currently possible to create two types of extensions for phpWatch: monitor extensions and channel extensions.  A
\emph{monitor extension} provides a new way of querying a service.  For example, one could write a monitor that interacts with
a service in a way unique to the specific service.  This can be extremely powerful for services that may require more
elaborate techniques to establish a connection or validate output.

\emph{Channel extensions} provide a new way of notifying contacts when a service is found to be offline or
non-functional.  phpWatch comes with a number of basic channels, namely one that e-mails contacts and one that can
text-message contacts.  By developing a channel extension, it is possible to notify humans in other ways or to react to
an outage automatically either with a PHP script alone or by invoking some outside process.

\section{Getting Started}
Developing extensions in phpWatch is designed to be simple but there is a slight learning curve.  To get started, it is
recommended that this document is read in its entirety.  Then, use the \texttt{ConnectionMonitor} as an example monitor
or \texttt{EmailChannel} as an example channel to learn from.  Both are relatively simple.

\section{Source Architecture}
phpWatch is built using common object-oriented practices with source divided into two directories.  The majority of the
logic resides in the \texttt{src} directory, within which are two more directories: \texttt{monitors} and
\texttt{channels}.  All extensions go into one of these two sub-directories and extend either the \texttt{Monitor} or
\texttt{Channel} class.

The second main source directory is \texttt{frontend}.  This houses all files related to the display of the phpWatch
frontend.  Extension developers need only be concerned with the \texttt{frontend/forms} directory.  Within this are two
more sub-directories: \texttt{monitors} and \texttt{channels}.  Much like the \texttt{src} directory, all extensions will
include code in one of these directories.

It is \emph{extremely} important to understand that some users run phpWatch without a frontend.  That is, they
manipulate the database through means other than the frontend.  Therefore it is crucial for extension developers to
understand which code goes within which directory.  \emph{All} code within the \texttt{src} directory must be entirely
void of HTML.  Nothing within this directory shall dictate the display capabilities of the system.  Conversely, as
little logic as possible should be placed in \texttt{frontend} to allow other users to implement their own frontend or
use scripts to manipulate information.

\section{Developing an Extension}
Extensions are comprised of \emph{logical} code and \emph{display} code.  Logical code handles the manipulation of data,
provides handles to the display code, and performs other actions specific to the extension.  For monitor extensions,
this code is placed in \texttt{src/monitors} and for channel extensions in \texttt{src/channels}.  Display code goes in
either \texttt{frontend/forms/monitors} or \texttt{frontend/forms/channels}.  Such code is comprised mainly of HTML and
function calls to some helper classes phpWatch provides.

\subsection{Monitors}
Monitor extensions provide a method of determining the status of services.  Specifically, it handles the querying of a
given service and discerning if the response indicates that the service is functioning properly (online) or improperly
(offline).

\subsubsection{Logic}
All monitors extend the \texttt{Monitor} class within \texttt{src}.  The new class \emph{must} be placed in
\texttt{src/monitors} in its own file named after the class.  For example, to create a monitor named ``ExampleMonitor'',
one would create the file \texttt{src/monitors/ExampleMonitor.php} containing a class \texttt{ExampleMonitor} which
extends \texttt{Monitor}.
 
By design, monitors rarely interact with the database.  This was done to reduce complexity and chances of corruption.
To maintain information about a monitor, the class variable \texttt{config} is used.  \texttt{\$this->config} is
accessible to monitor subclasses and is automatically saved to the database when the monitor is added or edited.  Thus,
if a monitor requires users to set a certain parameter, say a timeout period, the value of
    \texttt{\$this->config['timeout']} may be used.

Extending \texttt{Monitor} also requires that the developer implement at least the following methods:
\begin{center}
    \begin{tabular}{p{.25\textwidth}p{.5\textwidth}}
        \toprule
        \multicolumn{2}{l}{\texttt{public function queryMonitor()}} \\
        \midrule
        \multicolumn{2}{p{.8\textwidth}}{Handles the actual querying of the monitor.  This may entail opening a socket, sending a
        ping, etc.  The function must return \texttt{true} or \texttt{false} if the monitor is considered online or
        offline respectively.  This function is only called during cronjobs or manual re-querying.} \\
        \midrule
        \textbf{Parameters} & 
            \begin{tabular}{p{.5\textwidth}}
                None \\
            \end{tabular} \\ \midrule
        \textbf{Returns} &
             \begin{tabular}{p{.5\textwidth}}
                \texttt{true} if monitor is online \\
                \texttt{false} if monitor is offline \\
             \end{tabular} \\
        \bottomrule
    \end{tabular}
\\
    \begin{tabular}{p{.25\textwidth}p{.5\textwidth}}
        \toprule
        \multicolumn{2}{l}{\texttt{public function customProcessAddEdit(\$data, \$errors)}} \\
        \midrule
        \multicolumn{2}{p{.8\textwidth}}{This method is called when the form to add or edit a monitor is submitted.  The \texttt{\$data} array is simply
        the \texttt{\$\_POST} or \texttt{\$\_GET} array from the submission.  All fields added by this monitor shall be
        validated in this method.  If errors are encountered, append them to the \texttt{\$errors} array, keyed by
        form-field name with the value of an error message string.} \\
        \midrule
        \textbf{Parameters} & 
            \begin{tabular}{p{.5\textwidth}}
                \texttt{\$data}: Key-value array of form data. \\
                \texttt{\$errors}: Array of errors.  Append to this array if necessary. \\
            \end{tabular} \\ \midrule
        \textbf{Returns} &
             \begin{tabular}{p{.5\textwidth}}
                Possibly modified \texttt{\$errors} array.
            \end{tabular} \\
        \bottomrule
    \end{tabular}
\\
    \begin{tabular}{p{.25\textwidth}p{.5\textwidth}}
        \toprule
        \multicolumn{2}{l}{\texttt{public function customProcessDelete()}} \\
        \midrule
        \multicolumn{2}{p{.8\textwidth}}{This method is called when the form to delete a monitor is submitted.  It is
        the invoked before built-in phpWatch functions that delete any information from the database and can therefore
        function with full querying capabilities.} \\
        \midrule
        \textbf{Parameters} & 
            \begin{tabular}{p{.5\textwidth}}
                None
            \end{tabular} \\ \midrule
        \textbf{Returns} &
             \begin{tabular}{p{.5\textwidth}}
                None
             \end{tabular} \\
        \bottomrule
    \end{tabular}
\\
    \begin{tabular}{p{.25\textwidth}p{.5\textwidth}}
        \toprule
        \multicolumn{2}{l}{\texttt{public function getName()}} \\
        \midrule
        \multicolumn{2}{p{.8\textwidth}}{Gets the name of the monitor.  This should return a short string such as
        ``Connection Monitor'' as it's used to populate form field such as drop down menus.} \\
        \midrule
        \textbf{Parameters} & 
            \begin{tabular}{p{.5\textwidth}}
                None
            \end{tabular} \\ \midrule
        \textbf{Returns} &
             \begin{tabular}{p{.5\textwidth}}
                Short name of monitor as a string.
             \end{tabular} \\
        \bottomrule
    \end{tabular}
\\
    \begin{tabular}{p{.25\textwidth}p{.5\textwidth}}
        \toprule
        \multicolumn{2}{l}{\texttt{public function getDescription()}} \\
        \midrule
        \multicolumn{2}{p{.8\textwidth}}{Gets a description of the monitor.  This should return a string that describes
        how the monitor queries services.} \\
        \midrule
        \textbf{Parameters} & 
            \begin{tabular}{p{.5\textwidth}}
                None
            \end{tabular} \\ \midrule
        \textbf{Returns} &
             \begin{tabular}{p{.5\textwidth}}
                Description of monitor type.
             \end{tabular} \\
        \bottomrule
    \end{tabular}
\end{center}

\subsection{Channels}
Channel extensions provide a method of method of notification when a monitor is found to be offline.  More technically,
it dictates the set of actions to take when a monitor does not respond or malfunctions.

All channels extend the \texttt{Channel} class within \texttt{src}.  The new class \emph{must} be placed in
\texttt{src/channel} in its own file named after the class.  For example, to create a channel named ``ExampleChannel'',
one would create the file \texttt{src/channels/ExampleChannel.php} containing a class \texttt{ExampleChannel} which
extends \texttt{Channel}.
 
Like monitors, channels rarely interact with the database.  To maintain information about a channel, the class variable
\texttt{config} is used.  \texttt{\$this->config} is accessible to channel subclasses and is automatically saved to the
database when the channel is added or edited.  Thus, if a channel requires users to set a certain parameter, say an
e-mail address, the value of \texttt{\$this->config['email\_address']} may be used.

Additionally, extending \texttt{Monitor} requires that the developer implement at least the following methods:
\begin{center}
    \begin{tabular}{p{.25\textwidth}p{.5\textwidth}}
        \toprule
        \multicolumn{2}{l}{\texttt{public function doNotify()}} \\
        \midrule
        \multicolumn{2}{p{.8\textwidth}}{Called when a notification should be sent.  This method shall invoke custom
        code which sends a notification specific to the channel.} \\
        \midrule
        \textbf{Parameters} & 
            \begin{tabular}{p{.5\textwidth}}
                \texttt{\$monitor}: A reference to the monitor object that was found to be offline. \\
            \end{tabular} \\ \midrule
        \textbf{Returns} &
             \begin{tabular}{p{.5\textwidth}}
                None
             \end{tabular} \\
        \bottomrule
    \end{tabular}
\\
    \begin{tabular}{p{.25\textwidth}p{.5\textwidth}}
        \toprule
        \multicolumn{2}{l}{\texttt{public function customProcessAddEdit(\$data, \$errors)}} \\
        \midrule
        \multicolumn{2}{p{.8\textwidth}}{This method is called when the form to add or edit a channel is submitted.  The
        \texttt{\$data} array is simply the \texttt{\$\_POST} or \texttt{\$\_GET} array from the submission.  All fields
        added by this channel shall be validated in this method.  If errors are encountered, append them to the
        \texttt{\$errors} array, keyed by form-field name with the value of an error message string.} \\
        \midrule
        \textbf{Parameters} & 
            \begin{tabular}{p{.5\textwidth}}
                \texttt{\$data}: Key-value array of form data. \\
                \texttt{\$errors}: Array of errors.  Append to this array if necessary. \\
            \end{tabular} \\ \midrule
        \textbf{Returns} &
             \begin{tabular}{p{.5\textwidth}}
                Possibly modified \texttt{\$errors} array.
            \end{tabular} \\
        \bottomrule
    \end{tabular}
\\
    \begin{tabular}{p{.25\textwidth}p{.5\textwidth}}
        \toprule
        \multicolumn{2}{l}{\texttt{public function customProcessDelete()}} \\
        \midrule
        \multicolumn{2}{p{.8\textwidth}}{This method is called when the form to delete a channel is submitted.  It is
        the invoked before built-in phpWatch functions that delete any information from the database and can therefore
        function with full querying capabilities.} \\
        \midrule
        \textbf{Parameters} & 
            \begin{tabular}{p{.5\textwidth}}
                None
            \end{tabular} \\ \midrule
        \textbf{Returns} &
             \begin{tabular}{p{.5\textwidth}}
                None
             \end{tabular} \\
        \bottomrule
    \end{tabular}
\\
    \begin{tabular}{p{.25\textwidth}p{.5\textwidth}}
        \toprule
        \multicolumn{2}{l}{\texttt{public function getName()}} \\
        \midrule
        \multicolumn{2}{p{.8\textwidth}}{Gets the name of the channel.  This should return a short string such as
        ``E-mail Channel'' as it's used to populate form field such as drop down menus.} \\
        \midrule
        \textbf{Parameters} & 
            \begin{tabular}{p{.5\textwidth}}
                None
            \end{tabular} \\ \midrule
        \textbf{Returns} &
             \begin{tabular}{p{.5\textwidth}}
                Short name of channel as a string.
             \end{tabular} \\
        \bottomrule
    \end{tabular}
\\
    \begin{tabular}{p{.25\textwidth}p{.5\textwidth}}
        \toprule
        \multicolumn{2}{l}{\texttt{public function getDescription()}} \\
        \midrule
        \multicolumn{2}{p{.8\textwidth}}{Gets a description of the channel.  This should return a string that describes
        how the channel notifies contacts.} \\
        \midrule
        \textbf{Parameters} & 
            \begin{tabular}{p{.5\textwidth}}
                None
            \end{tabular} \\ \midrule
        \textbf{Returns} &
             \begin{tabular}{p{.5\textwidth}}
                Description of channel type.
             \end{tabular} \\
        \bottomrule
    \end{tabular}
\end{center}

\subsection{Display}
Display code for monitors and channels both consists of a form that allows for modification of \texttt{\$config} values.
As in the example above, if there is an entry in \texttt{\$this->config['email\_address']}, there will be a form field
for ``E-mail Address''.  It's important to note that fields common to all monitors (hostname, port, etc.) are generated
    automatically along with the form start/stop HTML.  Channels have no user-editable fields and therefore only the
    form start/stop HTML is automatically generated.

To create a form, a file named exactly the same as the monitor or channel must be placed in \\
\texttt{frontend/forms/monitor} or \texttt{frontend/forms/channel} respectively.  For example, to create a form for a
monitor named \texttt{ExampleMonitor} create the file \texttt{frontend/forms/monitor/ExampleMonitor.php}.  Field
generation is made simple with phpWatch.  The \texttt{FormHelpers} class provides methods to create most HTML field
types along with facilities for error handling.  A detailed summary of each method isn't shown but all fields except
select boxes can be created with

\begin{center}
    \texttt{FormHelpers::createTYPE(\$name, \$value, \$attribs = null)}
\end{center}

where \texttt{TYPE} should be replaced with a field type such as \texttt{Text}, \texttt{Hidden}, \texttt{Checkbox}, etc.

Select boxes can be created in a similar way with the only difference being the second parameter is now an array of
options, each generated with \texttt{FormHelpers::getOption(\$display, \$value, \$attribs = null)}.

See \texttt{frontent/FormHelpers.php} for more information on form creation and \\
\texttt{frontend/forms/monitors/ConnectionMonitor.php} for an example.

\section{Considerations}
Developing an extension for phpWatch was designed to be as simple as possible while still placing a great deal of
control in the hands of developers.  However, with this control comes the ability to cause error.  In particular,
monitor query code and channel notification code that fails either due to syntax error or exceptions will likely cause a
failure of cronjobs.

\section{When Not to Develop an Extension}
The developers of phpWatch want others to develop extensions.  In fact, the \emph{majority} of development time was
spent assuring others could quickly learn and integrate into phpWatch.  However, it is extremely important to realize
what should and what shouldn't be its own extension.

To make this decision, one must ask what is unique about the desired monitor or channel.  If you find yourself making
extensions that vary only in terms of parameters, that is a poor extension.  For example, making a monitor that waits 4
seconds for a server response and then making one that waits for a 10 second response is unnecessary.  Make a monitor
that waits $n$ seconds instead.

Although a trivial example, it is important to understand.  A good indicator of a poor monitor or channel is one with
nothing being set in the \texttt{\$this->configs} array.  This indicates that the monitor serves exactly one purpose
with no flexibility.  Although there are times this is acceptable, many times it is not.

\end{document}
